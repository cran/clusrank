%% The tex file is not to be edited; it is generated.
%% Only edit .Rnw file
%\VignetteIndexEntry{Clustered Rank-based Tests}
%\VignetteEngine{knitr::knitr}
\documentclass[nojss]{jss}\usepackage[]{graphicx}\usepackage[]{color}
%% maxwidth is the original width if it is less than linewidth
%% otherwise use linewidth (to make sure the graphics do not exceed the margin)
\makeatletter
\def\maxwidth{ %
  \ifdim\Gin@nat@width>\linewidth
    \linewidth
  \else
    \Gin@nat@width
  \fi
}
\makeatother

\definecolor{fgcolor}{rgb}{0.345, 0.345, 0.345}
\newcommand{\hlnum}[1]{\textcolor[rgb]{0.686,0.059,0.569}{#1}}%
\newcommand{\hlstr}[1]{\textcolor[rgb]{0.192,0.494,0.8}{#1}}%
\newcommand{\hlcom}[1]{\textcolor[rgb]{0.678,0.584,0.686}{\textit{#1}}}%
\newcommand{\hlopt}[1]{\textcolor[rgb]{0,0,0}{#1}}%
\newcommand{\hlstd}[1]{\textcolor[rgb]{0.345,0.345,0.345}{#1}}%
\newcommand{\hlkwa}[1]{\textcolor[rgb]{0.161,0.373,0.58}{\textbf{#1}}}%
\newcommand{\hlkwb}[1]{\textcolor[rgb]{0.69,0.353,0.396}{#1}}%
\newcommand{\hlkwc}[1]{\textcolor[rgb]{0.333,0.667,0.333}{#1}}%
\newcommand{\hlkwd}[1]{\textcolor[rgb]{0.737,0.353,0.396}{\textbf{#1}}}%

\usepackage{framed}
\makeatletter
\newenvironment{kframe}{%
 \def\at@end@of@kframe{}%
 \ifinner\ifhmode%
  \def\at@end@of@kframe{\end{minipage}}%
  \begin{minipage}{\columnwidth}%
 \fi\fi%
 \def\FrameCommand##1{\hskip\@totalleftmargin \hskip-\fboxsep
 \colorbox{shadecolor}{##1}\hskip-\fboxsep
     % There is no \\@totalrightmargin, so:
     \hskip-\linewidth \hskip-\@totalleftmargin \hskip\columnwidth}%
 \MakeFramed {\advance\hsize-\width
   \@totalleftmargin\z@ \linewidth\hsize
   \@setminipage}}%
 {\par\unskip\endMakeFramed%
 \at@end@of@kframe}
\makeatother

\definecolor{shadecolor}{rgb}{.97, .97, .97}
\definecolor{messagecolor}{rgb}{0, 0, 0}
\definecolor{warningcolor}{rgb}{1, 0, 1}
\definecolor{errorcolor}{rgb}{1, 0, 0}
\newenvironment{knitrout}{}{} % an empty environment to be redefined in TeX

\usepackage{alltt}

\usepackage[utf8]{inputenc}
\usepackage{amsmath}
\usepackage{amsfonts}
\usepackage{amssymb}
\usepackage{graphicx}
\usepackage{url}
\usepackage{natbib}
\usepackage{lmodern}
\usepackage{amssymb,amsmath}
\usepackage{multicol}
% use upquote if available, for straight quotes in verbatim environments
\IfFileExists{upquote.sty}{\usepackage{upquote}}{}
% use microtype if available
\IfFileExists{microtype.sty}{%
\usepackage{microtype}
\UseMicrotypeSet[protrusion]{basicmath} % disable protrusion for tt fonts
}{}
\usepackage[margin=1in]{geometry}
\usepackage{color}
\usepackage{fancyvrb}
\newcommand{\VerbBar}{|}
\newcommand{\VERB}{\Verb[commandchars=\\\{\}]}
\DefineVerbatimEnvironment{Highlighting}{Verbatim}{commandchars=\\\{\}}
% Add ',fontsize=\small' for more characters per line
\usepackage{framed}
\definecolor{shadecolor}{RGB}{248,248,248}
\newenvironment{Shaded}{\begin{snugshade}}{\end{snugshade}}
\newcommand{\KeywordTok}[1]{\textcolor[rgb]{0.13,0.29,0.53}{\textbf{{#1}}}}
\newcommand{\DataTypeTok}[1]{\textcolor[rgb]{0.13,0.29,0.53}{{#1}}}
\newcommand{\DecValTok}[1]{\textcolor[rgb]{0.00,0.00,0.81}{{#1}}}
\newcommand{\BaseNTok}[1]{\textcolor[rgb]{0.00,0.00,0.81}{{#1}}}
\newcommand{\FloatTok}[1]{\textcolor[rgb]{0.00,0.00,0.81}{{#1}}}
\newcommand{\CharTok}[1]{\textcolor[rgb]{0.31,0.60,0.02}{{#1}}}
\newcommand{\StringTok}[1]{\textcolor[rgb]{0.31,0.60,0.02}{{#1}}}
\newcommand{\CommentTok}[1]{\textcolor[rgb]{0.56,0.35,0.01}{\textit{{#1}}}}
\newcommand{\OtherTok}[1]{\textcolor[rgb]{0.56,0.35,0.01}{{#1}}}
\newcommand{\AlertTok}[1]{\textcolor[rgb]{0.94,0.16,0.16}{{#1}}}
\newcommand{\FunctionTok}[1]{\textcolor[rgb]{0.00,0.00,0.00}{{#1}}}
\newcommand{\RegionMarkerTok}[1]{{#1}}
\newcommand{\ErrorTok}[1]{\textbf{{#1}}}
\newcommand{\NormalTok}[1]{{#1}}

%%%%%%%%%%%%%%%%%%%%%%%%%%%%%%
%% declarations for jss.cls %%%%%%%%%%%%%%%%%%%%%%%%%%%%%%%%%%%%%%%%%%
%%%%%%%%%%%%%%%%%%%%%%%%%%%%%%
\author{\begin{tabular}{*{2}{>{\centering}p{.5\textwidth}}}
\large Nome1 & \large Nome2 \tabularnewline
Department1 & Department2 \tabularnewline
School1 & School2 \tabularnewline
\url{url1} & \url{url2}
\end{tabular}
}


\author{
Yujing Jiang\\ University of Connecticut \And
  Xin He\\ University of Maryland \And
  Mei-Ling Ting Lee\\ University of Maryland \AND
  Bernard Rosner\\ Harvard University  \And
  Jun Yan\\ University of Connecticut
  }
\title{Rank-Based Tests for Clustered Data with \proglang{R} Package
\pkg{clsrank}}

%% for pretty printing and a nice hypersummary also set:
\Plainauthor{Yujing Jiang, Mei-Ling Ting Lee, Jun Yan} %% comma-separated
\Plaintitle{R package Clustered Wilcoxon Rank Sum Test and Signed Rank Test} %% without formatting
\Shorttitle{\pkg{clsrank}: rank test for cluster data} %% a short title (if necessary)

%% an abstract and keywords


\Abstract{
Nonparametric tests such as Wilcoxon rank sum test
and Wilcoxon signed rank test are widely used in
the situation where the underlying distribution
of the population is far from normal or simply unknown.
One necessary assumption for the appropriateness of the
null distribution of the test statistic is that each
observation is independent, however, this is also an
assumption which is violated quite often in practice.
For instance, in the study of human eyes,
each person is the unit of
randomization, whereas the data is collected from both eyes,
therefore we sould expect correlation between the
data collected from the two eyes from the same person.
To account for the clustering effect, modifications of these
two tests have been proposed by \citet{Rosn:GLyn:Lee:inco:2003}
and \citet{Roxn:Glyn:Lee:wilc:2006}.
The modified tests work for both balanced and unbalanced data,
i.e., cluster size is idential or variable. In addition,
the modified rank sum test can also deal with stratified data.
No R package is available so far for nonparametric tests for
clustered data.
The package \pkg{clsrank} is a realization of the test
procedures from the two papers mentioned above with both
small and large sample tests.
}


\Keywords{Wilcoxon rank sum test, Wilcoxon signed rank test}
\Plainkeywords{Wilcoxon rank sum test, Wilcoxon signed rank test}

%% without formatting
%% at least one keyword must be supplied

%% publication information
%% NOTE: Typically, this can be left commented and will be filled out by the technical editor
%% \Volume{50}
%% \Issue{9}
%% \Month{June}
%% \Year{2012}
%% \Submitdate{2012-06-04}
%% \Acceptdate{2012-06-04}

%% The address of (at least) one author should be given
%% in the following format:

\Address{
Jun Yan\\
  Department of Statistics\\
  Professor of Statistics\\
  University of Connecticut\\
  215 Glenbrook Rd. Unit 4120, Storrs, CT 06269, USA.

  Email:\email{jun.yan@uconn.edu}
}
%% It is also possible to add a telephone and fax number
%% before the e-mail in the following format:
%% Telephone: +43/512/507-7103
%% Fax: +43/512/507-2851

%% for those who use Sweave please include the following line (with % symbols):
%% need no \usepackage{Sweave.sty}

%% end of declarations %%%%%%%%%%%%%%%%%%%%%%%%%%%%%%%%%%%%%%%%%%%%%%%




%% include your article here, just as usual
%% Note that you should use the \pkg{}, \proglang{} and \code{} commands.
\IfFileExists{upquote.sty}{\usepackage{upquote}}{}
\begin{document}
\section{Introduction}

Clustered data often arise in biomedical
studies, e.g., when the research objects are eyes,
ears, teath, etc. In these cases, the observations
can be classified into a number of distinct groups
or "clusters", where the observations are more similar
within each cluster than when they are from different clusters.
For instance, when measurements are taken from both eyes
of each patient, the measurments from the same person
should be correlated whereas measurements
taken from different patients should be independent.
Quite often, the goal of study is to compare the
measurements on the response variable
from the control group and the treatment group
or the measurements before and after a treatment
to see if the treatment effect is present.
The parametric statistical tests which explicitly account
for clustering include an adjusted version of the standard
two-sample t test which account for the intracluster correlation
(i.e., correlation among obervations within the clusters) when
the response variable follows a normal distribution and
a adjusted $\chi^2$ statistic when the response variable is binary
\citep{Donn:Bant:anal:1988}.


When the response variable is neither normal or binary,
e.g., the observations are ordinal,
the aforementioned tests are no longer suitable.
An attractive alternative is non-parametric test.
However, standard non-parametric tests requires the
observations to be independent, which is violated when the
data is clustered. If the intracluster correlation is
positive, then the variance of the standard test statistic
will be underestimated. \citet{Rosn:GLyn:Lee:inco:2003}
proposed a Wilcoxon rank sum test for clustered data
to compare two groups
assuming that the members within each cluster is
exchangeble and objects from the same cluster
belong to the same group.Their method is able
to deal with unbalanced data (i.e., data with
variable cluster sizes) and stratified data
(e.g., data from a multi-center study which
is stratified by centers).
Another modified
rank sum test for clustered data was proposed by
\citet{Datt:Satt:rank:2005}, their method is valid
even when the members of the same cluster came from
different groups, or when the intraclulster correlation
is determined by the group membership.
To compare paired observations, an adjusted version
of wilcoxon signed rank test was proposed by
\citet{Laro:wilc:2005} which invovles an estimate
of variance based on certain sum of square over
independent clusters, but this test procedure
is not distribution free.
\citet{Roxn:Glyn:Lee:wilc:2006} proposed another
modified signed rank test assuming a common
intercluster correlation and estimate it
from absolute rank of observations.
\citet{Datt:Satt:sign:2008} proposed a
signed rank test procedure based on the
principle of general with-in cluster resampling
and this test can handle the case when
the cluster size is informative, i.e.,
cluster size depends on the group.


Despite the popularity of clustered data in
a wide range of contexts such as
clinical trials, longitudinal study,
social science, etc,
there is no available R package nor function
publically available for
non-parametric test for clustered data yet.
Therefore a package pacted with
functions for this purpose will help
researchers and scientists
in related areas as a ready tool
for testing clustered data.
In this paper we present the \pkg{clusrank}
package, a
realization of test procedures presented
in \citet{Rosn:GLyn:Lee:inco:2003}
and \citet{Roxn:Glyn:Lee:wilc:2006}.
Large-sample inference based on asymptotic
distribution of test statistics and
small-sample inference based on permutation
are provided for both rank sum test and
signed rank test. The tests provided
can also handel unbalanced data when
there are clusters with different sizes,
in addition, for rank sum test, the
effect of
stratification as an extra cofounding
variable can also be accounted for.



%\textbf{Please, never hard code a reference.}
The order of this paper is as following,
section \ref{test} introduces Wilcoxon rank
sum test and Wilcoxon signed rank test
for clustered data, when data is balanced
or unbalanced. In addition, the modification
of rank sum test when data is stratified
is also discussed. Section \ref{real} is a real
data analysis for a eye study. The article
is summarized in section \ref{summ}.




\section{Tests}
\label{test}

%\textbf{The two tests can be introduced in one section
 % with subsections. Set up the problem first (with clustered data).
  %The finite sample permutation can be a subsection too.}

\subsection{Wilcoxan Rank Sum Test for Clustered Data}
\label{rank}

\subsubsection{Hypothesis}
\label{hrank}
Suppose there are two groups under different treatments,
$X$ and $Y$, the null hypothesis of the rank sum test
is that the probability of an observation from a
treatment $X$ exceeding an observation from treatment
$Y$ is the same as an observation from treatment $Y$
exceeding an observation from treatment $X$.
Specifically,
assume the data came in clusters and
let $X_{ij}$ denote the score for the $j$th subunit
from the $i$th cluster in the first group,
$i = 1,\ldots,m; j = 1,\ldots,g_i$ and $Y_{kl}$
denote the score for the $l$th subunit from
the $k$th cluster in the second group,
$k=1, \ldots,n;l=1,\ldots,h_k$.
The clustered Wilcoxon rank sum statistic
$W_{c,obs}$ is defined as
\begin{equation} \label{eq:Wco}
W_{c,obs} = \sum_{i=1}^m\sum_{j=1}^{g_i}\text{Rank}(X_{ij})
\end{equation}
where ranks are determined based on the
combined sample of all subunits over the
$X$ and $Y$ clusters. Subunits
for a given cluster are assumed to be exchangeable and each
cluster only contains members from one treatment group.


\subsubsection{Balanced data}\label{brank}
Under balanced designs, the cluster sizes are the same.
Since the score assigned to clusters under treatments
$X$ and $Y$ are
identically distributed under the null hypothesis,
we can pool $X$ and $Y$ clusters together and refer
to a combined set of $Z$ clusters, where $Z_{ij} = $score
for the $j$th subunit of the $i$th cluster,
$j = 1,\ldots,g, i=1,\ldots, m+n = N$.
Under $H_0$,
suppose that $m$ of the $N$ clusters are assigned
at random to the $X$ treatment and the remaining $n$
clusters to the $Y$ treatment.
Let $\delta_i$ be
an indicator function that equals $1$ when the $i$th cluster
is assigned to the $X$ group and $0$ when the $i$th cluster
is assigned to the $Y$ group.
The distribution of the clustered rank sum statistic $W_{c,obs}$ is
\begin{equation}\label{eq:Wc}
W_c = \sum^N_{i=1}\delta_iR_{i+} \quad \text{where   }R_{i+} = \sum^g_{j=1}R_{ij}
\end{equation}

where $R_{ij}=$rank of the $j$th subunit in the
$i$th cluster among all $gN$ subunits over all $Z$ clusters. It is shown that under $H_0$,
\begin{equation}\label{eq:EWc}
E(W_c) = gm(gN + 1)/2
\end{equation}
and
\begin{equation}\label{eq:VWc}
\text{Var}(W_c) = [mn/\{N(N-1)\}]\sum^N_{i=1}\{R_{i+} - g(1+gN)/2\}^2
\end{equation}
So a natural large sample test statistic based on \eqref{eq:Wc}, \eqref{eq:EWc}, and \eqref{eq:VWc} is
\begin{equation}
Z_c = \{W_c - E(W_c)\}/\{\text{Var}(W_c)\}^{1/2}
\end{equation}
$Z_c$ is asymptotically normal if both $m \to \infty$ and $n \to \infty$.
When the sample size is small, permutation test can be
performed.


\subsubsection{Unbalanced data}\label{ubrank}

For unbalanced designs, the null distribution
of the sum of rank
assigned to each cluster are not
identically distributed across clusters
with different sizes.
Let $(m_g, n_g)=$ number of clusters of
size $g$ assigned to the $X$ and $Y$ treatment
respectively.
Denote $N_g = m_g + n_g$ for $g=1,\ldots,g_{max}$
as the total number of clusters for clusters with
size $g$,
and $N =\sum_{g=1}^{g_{max}}N_g$ as the total number
of clusters in the sample.
Let $R_{ij,g} = $ rank for the $j$th
subunit in the $i$th cluster of size $g$,
$g = 1, \ldots, g_{max},
i = 1, \ldots, N_g, j = 1, \ldots, g$,
where $R_{ij,g}$s are computed based on the
combined sample.
The rank sum statistic can then be
written as:
\begin{equation}
W_{c, obs} = \sum_{g=1}^{g_{max}}
\sum_{i\in I_{g, obs}}R_{i+, g}
\end{equation}
where $R_{i+,g}$ = sum of ranks of
all subunits in the $i$th cluster of size $g$,
$i = 1, \ldots, N_g$.
The distribution corresponding to $W_{c, obs}$
is
\begin{equation}
\label{eq:ubrankstat}
W_c = = \sum_{g = 1}^{g_{max}}
 = \sum_{g = 1}^{g_max}\sum_{i = 1}^{N_g}\delta_{i, g}R_{i+, g}
\end{equation}
where $\delta_{i, g} = 1$ if the $i$th
cluster of size $g$ is assigned to group $X$, and is $0$
otherwise.
The corresponding $E(W_c)$ and $\text{Var}(W_c)$ are
as following:
\begin{equation}\label{eq:uEWc}
E(W_c) = \sum_{g=1}^{g_{max}}m_g(R_{++,g}/N_g)
\end{equation}
\begin{equation}\label{eq:uVWc}
\text{Var}(W_c) = \sum_{g=1}^{g_{max}}[m_gn_g/\{N_g(N_g - 1)\}]\sum_{i=1}^{N_g}(R_{i+,g} - R_{++,g}/N_g)^2
\end{equation}
A large-sample test statistics based on $W_c$ is as following:
\begin{equation}
Z_c = \{W_c - E(W_c)\}/\{\text{Var}(W_c)\}^{1/2}
\end{equation}
Again, when sample size is small, permutation test can
be applied based on \eqref{eq:ubrankstat}

When applying the ranksum test for the clustered data,
a formula is used as a interface where the response
variable is on the righthand side and covariates on
the lefthand side. Special functions \code{group},
\code{cluster} and \code{stratum} are used to indicate
the function of each variable, \code{group} indicates
the id of the treatment group, \code{cluster} indicates
the membership of an observation in a specific cluster
and \code{stratum} indicates the stratum an observation
belongs to.

\begin{knitrout}
\definecolor{shadecolor}{rgb}{0.969, 0.969, 0.969}\color{fgcolor}\begin{kframe}
\begin{alltt}
\hlkwd{library}\hlstd{(clusrank)}
\hlkwd{data}\hlstd{(crd)}
\hlcom{## Using large-sample test.}
\hlkwd{cluswilcox.test}\hlstd{(z} \hlopt{~} \hlkwd{group}\hlstd{(group)} \hlopt{+} \hlkwd{cluster}\hlstd{(id),} \hlkwc{data} \hlstd{= crd)}
\end{alltt}
\begin{verbatim}
## 
## 	Wilcoxon rank sum test for clutered data
## 
## data:  z from crd, cluster: id, group: group,
## Rank sum statistic = 19505
## Expected value of rank sum statistic = 20013
## Variance of rank sum statistic = 631030
## Test statistic = -0.63924, p-value = 0.5227
## difference in locations = 0
## The data is unbalanced
## alternative hypothesis: true difference in locations is not equal to 0
\end{verbatim}
\begin{alltt}
\hlcom{## For small sample, using the permutation test.}
\hlkwd{cluswilcox.test}\hlstd{(z} \hlopt{~} \hlkwd{group}\hlstd{(group)} \hlopt{+} \hlkwd{cluster}\hlstd{(id),} \hlkwc{data} \hlstd{= crd,} \hlkwc{permutation} \hlstd{=} \hlnum{TRUE}\hlstd{)}
\end{alltt}
\begin{verbatim}
## 
## 	Wilcoxon rank sum test for clutered data
## 
## data:  z from crd, cluster: id, group: group,
## Rank sum statistic = 19505
## p-value = 0.536
## location = 0
## The data is unbalanced
## alternative hypothesis: true location is not equal to 0
\end{verbatim}
\end{kframe}
\end{knitrout}


\subsubsection{Data with Stratification}
Futher more, if the data is also stratified,
the test statistic and its null distribution will
need modification to control for stratification as
an extra
confounding variable.
Suppose there are $V$ strata,
let $(m_{g,v}, n_{g,v})=$ number of clusters of size $g$
in stratum $v$ assigned to the$X$ and $Y$ clusters of size $g$
in stratum $v$ respectively,
$g = 1, \ldots, g_{max}, v = 1, \ldots, V$.
Let $R_{i+, g, v}$ be the rank sum for the subunits in the
$i$th cluster of size $g$ in the $v$th stratum.
The rank sum statistic is defined as

\begin{equation}
\label{eq: srank}
W_{c, obs} = \sum_{g=1}^{g_{max}}\sum_{v=1}^{V}\left(\sum_{i\in I_{g,v,obs}}R_{i+,g,v}\right)
\end{equation}
where $I_{g, v, obs}$ is the observed subset of $m_{g,v}$
unique indices selected from
$\{1, \ldots, N_{g,v}\}$,
corresponding to cluster of
size $g$ in stratum $v$ that are assigned to the $X$ treatment.
The corresponding expectation and variance of the null
distribution of $W_{c, obs}$

\begin{align*}
E(W_c)=& \sum_{g=1}^{g_{max}}\sum^V_{v=1}m_{g,v}R_{++,g,v}/N_{g,v}\\
\text{Var}(W_c)=& \sum_{g=1}^{g_{max}}\sum_{v=1}^V[m_{g,v}n_{g,v}/\{N_{g,v}(N_{g,v} - 1)\}]\\
&\times\sum_{i=1}^{N_{g,v}}(R_{i+, g, v} - R_{++, g, v}/N_{g,v})^2
\end{align*}

The large sample test statistic is then
\begin{equation}
Z_c = \{W_c - E(W_c)\}/\{\text{Var}(W_c)\}^{1/2}.
\end{equation}
Again, permutation test can be applied when sample size is small.

Following is an illustration of the rank sum test for the
stratified data, the \code{crdStr} data is a test data set
comes with the package:
\begin{knitrout}
\definecolor{shadecolor}{rgb}{0.969, 0.969, 0.969}\color{fgcolor}\begin{kframe}
\begin{alltt}
\hlkwd{data}\hlstd{(crdStr)}
\hlkwd{cluswilcox.test}\hlstd{(z} \hlopt{~} \hlkwd{group}\hlstd{(group)} \hlopt{+} \hlkwd{cluster}\hlstd{(id)} \hlopt{+} \hlkwd{stratum}\hlstd{(stratum),} \hlkwc{data} \hlstd{= crdStr)}
\end{alltt}
\begin{verbatim}
## 
## 	Wilcoxon rank sum test for clutered data
## 
## data:  z from crdStr, stratum: stratum, cluster: id, group: group,
## Rank sum statistic = 222250
## Expected value of rank sum statistic = 227660
## Variance of rank sum statistic = 20338000
## Test statistic = -1.198, p-value = 0.2309
## difference in locations = 0
## 
## alternative hypothesis: true difference in locations is not equal to 0
\end{verbatim}
\end{kframe}
\end{knitrout}


\subsection{Wilcoxon signed rank test for clustered data}
\label{sign}
\subsubsection{Hypothesis}
\label{hsign}
Let $X_{ij} (Y_{ij})$ denotes the baseline
(follow-up) score for the $j$th subunit
in the $i$th cluster (subject) and
define $Z_{ij} = Y_{ij} - X_{ij}, j = 1,
\ldots,g_i; i = 1,\ldots,m$.
Within each cluster, the difference
scores are assumed to be independent and
identically distributed. The signed rank
test is used to find out if the population
is shifted after the treatment.
Formally, hypothesis being tested is

\begin{equation*}
H_0: \text{the difference score } Z \text{ is symmetric about 0}
\end{equation*}
vs
\begin{equation*}
H_1: Z \text{ is symmetric about }\gamma, \gamma \not = 0.
\end{equation*}

Rank $|Z_{ij}|$ over the total of
$G = \sum_{i=1}^m g_i$ subunits from
the $m$ clusters and let $S_{ij}=R_{ij}V_{ij}$,
where $R_{ij} = $ rank of $|Z_{ij}|$
within the total data set of $G$
subunits over $m$ clusters, and $V_{ij} = \text{sign}(Z_{ij})$.


\subsubsection{Balanced Data}
\label{bsign}
If the data is balanced,
the clustered
Wilcoxon signed rank statistic is defined as following:

\begin{equation}
T_{c}^{(obs)} = \sum_{i=1}^mS_{i+} \equiv \sum^m_{i=1}\sum^g_{j=1}R_{ij}V_{ij},
\end{equation}


where $S_{i+} = \sum_{j=1}^gS_{ij}$ which
is the sum of the rank within $i^{th}$ cluster
and only consider nonzero
$Z_{ij}$ in the computation of signed ranks.
When considering the randomization
distribution corresponding to $T_c$,
the unit of randomization is the cluster.
Let $\delta_1, \ldots,\delta_m$ be i.i.d.
random variables each taking on the values
+1 and -1 with probability 1/2, then the
distribution of $T_{c}^{(obs)}$ is:

\begin{equation} \label{eq:tc}
T_c = \sum_{i=1}^m\delta_iS_{i+}.
\end{equation}

It is shown that under $H_0$, $E(T_c) = 0$
and $\text{Var}(T_c) = \sum^m_{i=1}S^2_{i+}$.
Standarize $T_{c}^{(obs)}$ with $E(T_c)$ and
$\text{Var}(T_c)$,
the large sample test statistic is defined as
\begin{equation}
Z_c = \left. T_c \middle/ \left(\sum_{i=1}^nS^2_{i+}\right)^{1/2}\right. \sim N(0, 1)\qquad \text{under } H_0.
\end{equation}
When the sample size is small, a permutation
test can be applied.


For the signed rank test, the input should be a numeric
vector which contains the difference between the paired
observations, or two vectors, where vector
\code{x} contains observations before the treatment,
and \code{y} contains observations after the treatment.
This manner of input is restricted to the signed rank
test.

\begin{knitrout}
\definecolor{shadecolor}{rgb}{0.969, 0.969, 0.969}\color{fgcolor}\begin{kframe}
\begin{alltt}
\hlcom{## Large sample signed rank test for clustered data}
\hlkwd{data}\hlstd{(crsd)}
\hlkwd{cluswilcox.test}\hlstd{(z,} \hlkwc{cluster} \hlstd{= id,} \hlkwc{data} \hlstd{= crsd)}
\end{alltt}
\begin{verbatim}
## 
## 	Wilcoxon signed rank test for clutered data
## 
## data:  z, cluster: id from crsd
## rank statistic = -110
## Variance of  rank statistic = 30178
## test statistic = -0.63321, p-value = 0.5266
## total number of observations = 40, total number of clusters = 20
## 
## alternative hypothesis: true location shift is not equal to 0
\end{verbatim}
\begin{alltt}
\hlcom{## Small sample test}
\hlkwd{data}\hlstd{(crsd)}
\hlkwd{cluswilcox.test}\hlstd{(z,} \hlkwc{cluster} \hlstd{= id,} \hlkwc{data} \hlstd{= crsd,} \hlkwc{permutation} \hlstd{=} \hlnum{TRUE}\hlstd{)}
\end{alltt}
\begin{verbatim}
## 
## 	Wilcoxon signed rank test for clutered data using permutation
## 
## data:  z, cluster: id from crsd
## rank statistic = -110
## p-value = 0.524
## total number of observations = 40, total number of clusters = 20
## 
## alternative hypothesis: true location shift is not equal to 0
\end{verbatim}
\end{kframe}
\end{knitrout}

\subsubsection{Unbalanced Data}
\label{ubsign}
When the data is unbalanced
the test statistic is defined as:

\begin{equation}
T_c^{(obs)} = \sum_{i = 1}^m w_i\bar{S}_i
\end{equation}

where $\bar{S}_i = S_{i+}/{g_i}, w_i = 1/\text{Var}(\bar{S}_i)$ under $H_0$.
The randomization distribution
corresponding to
$T_{c,s}^{obs} = \sum^m_{i=1}\delta_iw_i\bar{S}_i$.
$\delta$ is defined as in \eqref{eq:tc}.
The test statistic is defined as
\begin{equation}
Z_{c,s} = \left.T_{c,s}\middle/ \left(\sum_{i=1}^m\hat{w}^2_i\bar{S}^2_i\right)^{1/2}\right. \sim N(0,1) \text{ under }H_0,
\end{equation}

where $\hat{w}_i = g_i/[\widehat{Var}(S_{ij})\{1 + (g_i-1)\hat{\rho}_{s,cor}\}]$, $\hat{\rho}_{s,cor} = \hat{\rho}_s\left(1 + \frac{1 - \hat{\rho}_s^2}{m - 5/2}\right)$,$\hat{\rho_s} = $ max $[\hat{\sigma}_A^2/(\hat{\sigma}_A^2 + \hat{\sigma}^2), 0]$, $\hat{\sigma
}^2 = \sum_{i=1}^m\sum_{j=1}^{g_i}(S_{ij} - \bar{S}_i)^2/(G-m), \hat{\sigma}^2_A = $ max $[\{\sum_{i=1}^mg_i(\bar{S_i}- \bar{\bar{S}})^2/(m-1)- \hat{\sigma}^2\}/g_0,0], g_0 = [\sum_{i=1}^mg_i - \sum_{i=1}^mg_i^2/\sum_{i=1}^mg_i]/(m-1)$ and $\widehat{Var}(S_{ij}) = \sum_{i=1}^m\sum_{j=1}^{g_i}(S_{ij} - \bar{\bar{S}})^2/(G-1)$.

An illustration of use of the test is as following:
\begin{knitrout}
\definecolor{shadecolor}{rgb}{0.969, 0.969, 0.969}\color{fgcolor}\begin{kframe}
\begin{alltt}
\hlkwd{data}\hlstd{(crsdUnb)}
\hlkwd{cluswilcox.test}\hlstd{(z,} \hlkwc{cluster} \hlstd{= id,} \hlkwc{data} \hlstd{= crsdUnb)}
\end{alltt}
\begin{verbatim}
## 
## 	Wilcoxon signed rank test for clutered data
## 
## data:  z, cluster: id from crsdUnb
## adjusted rank statistic = -0.015709
## Variance of  adjusted rank statistic = 0.00123
## test statistic = -0.44794, p-value = 0.6542
## total number of observations = 748, total number of clusters = 142
## The signed rank test statistics is adjusted since the data is unbalanced.
## alternative hypothesis: true location shift is not equal to 0
\end{verbatim}
\end{kframe}
\end{knitrout}



\section{Real Data Analysis}
\label{real}

\subsection{Data}
To illustrate the usage of the package, we are going to
perform the clustered Wilcoxon rank sum test on a
real data set in a study of eyes.
Age-related macular degeneration (AMD) is a disease
that blurs the sharp centeral vision by
affecting macula, a oval yellow spot
near the center of the retina of the human eye.
The complement factor H R1210C is a large protein
that circulates in human plasma. The variant of
this protein confers the strongest genetic risk
for AMD and earlier age at onset. The objective
of the study is to characterize the observable
traits of this variant. The study was carried out by
the Seddon Lab \citep{Sedd:Shar:Adel:eval:2006,
Ferr:Sedd:phen:2015} with 143 patients (283 eyes)
involved, including 62 patients with the rare
variant. The degree of severity of AMD was graded
based on the Clinical
Age-Related Maculopathy Staging (CARMS) system for
each enrolled eye.
The CARMS system has a
5-step scale, where 1 to 3 represent no symptom,
earlier and intermediate severity respectively,
4 and 5 represent
two different symptoms of the advanced AMD,
geographic atrophy and neovascular disease respectively.
The
CARMS grades were assessed separately for these
two advanced stages, i.e., we are going
to carry out the analysis on two subsets of the
observations respectively: the subset of
observations with
CARMS grade 1, 2, 3, or 4, and the subset of
observations with CARMS grade 1, 2, 3, or 5.
Since high correlation between
eyes of the same patient is expected while
observations from different patients can be assumed
as independent, the data is clustered in pairs and
each subject is a cluster. The data also provided
information on age and sex, and an extra variable
combined age and sex, which could be used to stratify
the data.
In this analysis, the CARMS grade is the response
variable and treatment refers to the presence
of the complement factor H R1210C variant.
The data set contains 7 variables: ID is the
subject identifier which is the cluster id.


\begin{knitrout}
\definecolor{shadecolor}{rgb}{0.969, 0.969, 0.969}\color{fgcolor}\begin{kframe}
\begin{alltt}
\hlcom{## Carry out clustered rank sum test for the subset}
\hlcom{## with CARMS grade 1, 2, 3 and 4.}
\hlkwd{data}\hlstd{(sedlab)}
\hlkwd{cluswilcox.test}\hlstd{(CARMS} \hlopt{~} \hlkwd{cluster}\hlstd{(ID)} \hlopt{+} \hlkwd{stratum}\hlstd{(Agesex)} \hlopt{+} \hlkwd{group}\hlstd{(Variant),}
                \hlkwc{data} \hlstd{= sedlab,} \hlkwc{subset} \hlstd{= CARMS} \hlopt \hlkwd{c}\hlstd{(}\hlnum{1}\hlstd{,} \hlnum{2}\hlstd{,} \hlnum{3}\hlstd{,} \hlnum{4}\hlstd{))}
\end{alltt}
\begin{verbatim}
## 
## 	Wilcoxon rank sum test for clutered data
## 
## data:  CARMS from sedlab, stratum: Agesex, cluster: ID, group: Variant,
## Rank sum statistic = 8792
## Expected value of rank sum statistic = 10867
## Variance of rank sum statistic = 258650
## Test statistic = -4.0797, p-value = 4.509e-05
## difference in locations = 0
## The data is unbalanced
## alternative hypothesis: true difference in locations is not equal to 0
\end{verbatim}
\end{kframe}
\end{knitrout}

\begin{knitrout}
\definecolor{shadecolor}{rgb}{0.969, 0.969, 0.969}\color{fgcolor}\begin{kframe}
\begin{alltt}
\hlcom{## Carry out clustered rank sum test for the subset}
\hlcom{## with CARMS grade 1, 2, 3 and 5.}
\hlkwd{data}\hlstd{(sedlab)}
\hlkwd{cluswilcox.test}\hlstd{(CARMS} \hlopt{~} \hlkwd{cluster}\hlstd{(ID)} \hlopt{+} \hlkwd{stratum}\hlstd{(Agesex)} \hlopt{+} \hlkwd{group}\hlstd{(Variant),}
                \hlkwc{data} \hlstd{= sedlab,} \hlkwc{subset} \hlstd{= CARMS} \hlopt \hlkwd{c}\hlstd{(}\hlnum{1}\hlstd{,} \hlnum{2}\hlstd{,} \hlnum{3}\hlstd{,} \hlnum{5}\hlstd{))}
\end{alltt}
\begin{verbatim}
## 
## 	Wilcoxon rank sum test for clutered data
## 
## data:  CARMS from sedlab, stratum: Agesex, cluster: ID, group: Variant,
## Rank sum statistic = 14502
## Expected value of rank sum statistic = 15583
## Variance of rank sum statistic = 340860
## Test statistic = -1.8519, p-value = 0.06404
## difference in locations = 0
## The data is unbalanced
## alternative hypothesis: true difference in locations is not equal to 0
\end{verbatim}
\end{kframe}
\end{knitrout}

When controlled for the stratification covariate \code{Agesex},
the p-value for CARMS grades 1-4 is less than $0.001$,
which implies strong correlation between the
presence of the complement factor H R1210C variant and
the severity of AMD when treating geographic atrophy as the
advanced stage. The p-value for CARMS grades 1,2,3 and 5
is 0.06, again the evidence is relatively strong the
the presence of the variant does affect the severity of
AMD when treating neovascular disease as the advanced stage.





\section{Summary}
\label{summ}
In this artical, Wilcoxon rank sum test and Wilcoxon signed rank
test adjusted for the clustering effect in the data are
introduced. The usage of the \proglang{R} package \pkg{clusrank} carrying out the
two tests is illustrated with examples.
Both the tests are able to handel unbalanced data.
In addition, the clustered Wilcoxon rank sum test also permits an
extra covariate as stratification variable.
\section*{Acknowledgement}
Data \textbf{sedlab} is provided by Seddon Lab
\citep{Sedd:Shar:Adel:eval:2006,
Ferr:Sedd:phen:2015}.
%% Note: If there is markup in \(sub)section, then it has to be escape as above.


%\bibliographystyle{jss}

\bibliography{clusrank}

\end{document}
